%----------------------- Wydruk dwustronny ---------------
%\documentclass[12pt,twoside,a4paper]{book} % 
%----------------------- Wydruk jednostronny ---------------
\documentclass[12pt,oneside,a4paper]{book} % jednostronnego


\usepackage{polski}
\usepackage[utf8]{inputenc} %opcja dla edytorów kodujących polskie znaki w utf8
%\usepackage[cp1250]{inputenc} %opcja dla edytorów kodujących polskie znaki w windows-1250
\usepackage{lmodern}
\usepackage{indentfirst}
\usepackage{microtype}
\DisableLigatures{encoding = *, family = * }
\usepackage{fancyhdr}
\usepackage{pstricks,graphicx}
\usepackage{amssymb}
\def\bibname{Literatura}% zmienia nazwę Bibliografia na literatura
\usepackage{changepage}% zmieni pozycję marginesu


%---------------Zbiory liczbowe
\newcommand{\R}{\mathbb{R}}
\newcommand{\N}{\mathbb{N}}
\newcommand{\K}{\mathbb{K}}
\newcommand{\C}{\mathcal{C}}
\newcommand{\p}{\mathcal{P}}
%------------kwantyfikatory--------------
\newcommand{\fal}{\mbox{{\Large $\forall\,$}}}
\newcommand{\ext}{\mbox{{\Large $\exists\,$}}}
%------------------definicje środowisk-----------------
\usepackage{theorem}
\theoremstyle{break}
\theorembodyfont{\it}
\newtheorem{twr}{Twierdzenie}[chapter]
\newtheorem{lem}{Lemat}[chapter]
\theorembodyfont{\rm}
\newtheorem{defi}{Definicja}[chapter]
\newtheorem{wni}{Wniosek}[chapter]
\newtheorem{prz}{Przykład}[chapter]
\newenvironment{dowod}{\par\vspace{0.1cm}\par{ \sc Dowód.}}{\hfill $\blacksquare$\par\vspace{0.4cm}\par}
% ----------ustawienia wymiarow strony
\usepackage{geometry}

\newgeometry{tmargin=2.5cm, bmargin=2.5cm, headheight=14.5pt, inner=3cm, outer=2.5cm} 

\linespread{1.1} %-zmiana interlinii

%---------------- Normalne środowiska --------------------
\usepackage{amsmath}

%----------nagłowki i żywa pagina ------------
\pagestyle{fancy} 
%--------------- Wydruk dwustronny
%\cfoot[]{} 
%\lhead[{\scriptsize{\it \thepage}}]{}
%\chead[{\scriptsize\leftmark}]{{\scriptsize \rightmark}}
%\rhead[]{{\scriptsize{\it \thepage}}}
%--------------- Wydruk jednostronny
\fancyhead[C]{} 
\fancyfoot[C]{\thepage}
\fancyhead[L]{\scriptsize\leftmark}
\fancyhead[R]{\scriptsize\rightmark}

\renewcommand{\chaptermark}[1]{%
\markboth{\MakeUppercase{%
\chaptername}\ \thechapter.%
\ #1}{}}

\renewcommand{\sectionmark}[1]{\markright{\thesection.\ #1}}


\usepackage{lipsum} %------ Można usunąć przy pisaniu pracy
%-----------------właściwa część pracy-----------------
\begin{document}
\thispagestyle{empty}

\begin{center}{\sc \Large
Uniwersytet Gdański\\
Wydział Matematyki, Fizyki i Informatyki
}
\end{center}

\vspace{3cm}

\begin{adjustwidth}{120pt}{0pt}
\begin{flushleft}
{\large \textbf{Agnieszka Kamińska\\
Aleksandra Panek\\
Karolina Pomian\\
}
}\par\vspace{0.5cm}\par
{\large
Kierunek: Informatyka Ogólnoakademicka\\
Specjalność: \\
Numery albumu: x, x, 275552\\
}
\end{flushleft}
\end{adjustwidth}

\vspace{1cm}

\begin{center}{ \Large \textbf {
Tytuł pracy
}}
\end{center}

\vspace{3cm}

\begin{flushright}
%praca dyplomowa\\
Praca licencjacka napisana\\
pod kierunkiem…………\\
Hanna Furmańczyk
\end{flushright}
\vfill
\begin{center}{\large
\textbf{Gdańsk 2023}}
\end{center}

\newpage

\vspace*{16cm}
\begin{flushright}
miejsce na ewentualne podziękowania\\
dla promotora
\end{flushright}



\thispagestyle{empty} \setcounter{page}{0} \tableofcontents

\chapter*{Wstęp}

\thispagestyle{empty} \addcontentsline{toc}{chapter}{Wstęp}

Wstęp do pracy dyplomowej. Zwykle pisany na końcu i zawierający streszczenie i podsumowanie najważniejszych aspektów i celów pracy.
\chapter{Pojęcia wstępne}
\thispagestyle{empty}

\pagestyle{fancy}




\lipsum[1-2]
\section{Definicje}
\[
\fal x\in X \quad \qquad \mbox{dla każdego }
\]
\begin{defi}
Niech $x\in X$ będzie dowolnie ustalone.
%punktory
\begin{enumerate} %-- różne style punktorów i numeratorów
\item punktor pierwszy
\item punktor 2
\item[{\rm (i)}] $x >y \Leftrightarrow x-y>0$
\item[1$^o$] $x >y \Leftrightarrow x-y>0$
\end{enumerate}
\end{defi}

\section{twierdzenia}
W pracy \cite{poz23} można znaleźć %----- cytowanie literatury
\begin{twr}[Pitagoras]
\label{t1}%etykieta twierdzenia
Suma kwadratów przyprostokątnych w trójkącie prostokątnym jest równa kwadratowi przeciwprostokątnej tego trójkąta
\end{twr}
\begin{dowod}
Niech $a$, $b$ i $c$ oznaczają .....
\end{dowod}
Odwołoanie do twierdzenia: Z twierdzenia \ref{t1} wynika
\begin{prz}
Niech $(X,d)$ będzie przestrzenią metryczną. ...
\end{prz}


\lipsum
\lipsum


\chapter{tytuł rozdziału drugiego}
\thispagestyle{empty}

\lipsum[1-2]




\section{wzory wyróżnione nienumerowane}
\subsection{Wzór klasyczny}
Na dwa sposoby:
\[
\lim\limits_{n\to \infty}n=\sup \N=\{1,2,\ldots \}
\]
$$\lim\limits_{n\to \infty}n=\sup \N=\{1,2,\ldots \}$$
\subsection{Wzór wieloliniowy z wyrównaniem}
\begin{align*}
f(x) &= x^2\\
& {} + xy^3\\
& = 0
\end{align*}
Szersze odstępy  %--- 5ex kontroluje szerokość, można dostosować
\begin{align*} 
f(x)=&\\[5ex]  
&  jkhdhkdfhkjdsfhkjsfhksdhfskhgfkshfksgh\\[5ex]
& {} + hsdkjhfgkjhgkjhfgkjhgfkj
\end{align*}
\section{wzory wyróżnione numerowane}
Pojedyczny wzór numerowany:
\begin{equation}
\label{wz345}%etykieta wzoru
x+y=y+x \qquad \mbox{ dla dowolnych }x,y\in X
\end{equation}
Wzory wieloliniowe z opcją numerowania tylko wybranych linii:
\begin{align}
\nonumber f(x) &= ax^2+bx\\
\label{wz2} & {} + c\\
\label{wz4} & = a(x-x_0)(x-x_1)
\end{align}
Wzór wieloliowy z globalnym numerem:
\begin{equation}\label{wz3}
\begin{array}{rcl}
f(x)& =& ax+b\\[3ex]
g(x)&=&\frac{ax+b}{cx+d}
\end{array}
\end{equation}
Równość (\ref{wz345}) implikuje .....


\begin{thebibliography}{00}
\addcontentsline{toc}{chapter}{Literatura}

%------------- Książka ---------------------
\bibitem{day}
M.M. Day, \textit{Normed Linear Spaces}, Springer-Verlag, Berlin
-- Heidelberg -- New York, 1973.

%------------- Artykuł ---------------------
\bibitem{james} R.C. James, \textit{Orthogonality in normed linear linear spaces}, Duke Math. J., {\bf
12} (1945), 291--301.

\bibitem{k} M. Kuczma, \textit{An Introduction to the
Theory of Functional Equations and Inequalities}, PWN -
Uniwersytet Śląski, Warszawa - Kraków - Katowice, 1985.

%------------- Skrypty, notatki do wykładu ---------------------
\bibitem{poz23}
W. Pleśniak, \  \textit{Wykłady z teorii aproksymacji}, WUJ,
Kraków 2000.

%------------- Strona internetowa ---------------------
\bibitem{OEF}
Numbers of edges of regular polygons constructible with ruler and compass (A003401), The on-line encyclopedia of integer sequences, https://oeis.org/A003401 

\end{thebibliography}
\newpage\thispagestyle{empty}
Imię i Nazwisko \hfill Kraków, dnia wpisać date\par\vspace{1cm}\par
\centerline{{\Large \bf O Ś W I A D C Z E N I E}}
\par\vspace{1.4cm}\par
Świadomy(a) odpowiedzialności oświadczam, że przedłożona praca pt.:
\begin{center}
{\sc Tu wpisać tytuł pracy}
\end{center}
została napisana przeze mnie samodzielnie.
\par
Jednocześnie oświadczam że w/w praca nie narusza praw autorskich w rozumieniu Ustawy z dnia 4 lutego 1994 r. o prawie autorskim i prawach pokrewnych (Dz.U. z 2006 nr 90, poz.
631 z późn. zmianami) oraz dóbr osobistych chronionych prawem cywilnym.
\par
Przedłożona praca nie zawiera danych empirycznych ani też informacji, które uzyskałem(am) w sposób niedozwolony. Stwierdzam, iż przedstawiona praca w całości ani też w części nie była wcześniej podstawą żadnej innej urzędowej procedury związanej z~uzyskaniem dyplomu ani też nadania tytułów zawodowych.
\par\vspace{1cm}\par
\begin{flushright}
....................................... \ \ \ \ \ \ \\
{\scriptsize (podpis)\hspace{2.3cm}\ }
\end{flushright}


\end{document}