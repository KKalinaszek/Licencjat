%----------------------- Wydruk dwustronny ---------------
%\documentclass[12pt,twoside,a4paper]{book} % 
%----------------------- Wydruk jednostronny ---------------
\documentclass[12pt,oneside,a4paper]{book} % jednostronnego


\usepackage{polski}
\usepackage[utf8]{inputenc} %opcja dla edytorów kodujących polskie znaki w utf8
%\usepackage[cp1250]{inputenc} %opcja dla edytorów kodujących polskie znaki w windows-1250
\usepackage{lmodern}
\usepackage{indentfirst}
\usepackage{microtype}
\DisableLigatures{encoding = *, family = * }
\usepackage{fancyhdr}
\usepackage{pstricks,graphicx}
\usepackage{amssymb}
\def\bibname{Literatura}% zmienia nazwę Bibliografia na literatura
\usepackage{changepage}% zmieni pozycję marginesu


%---------------Zbiory liczbowe
\newcommand{\R}{\mathbb{R}}
\newcommand{\N}{\mathbb{N}}
\newcommand{\K}{\mathbb{K}}
\newcommand{\C}{\mathcal{C}}
\newcommand{\p}{\mathcal{P}}
%------------kwantyfikatory--------------
\newcommand{\fal}{\mbox{{\Large $\forall\,$}}}
\newcommand{\ext}{\mbox{{\Large $\exists\,$}}}
%------------------definicje środowisk-----------------
\usepackage{theorem}
\theoremstyle{break}
\theorembodyfont{\it}
\newtheorem{twr}{Twierdzenie}[chapter]
\newtheorem{lem}{Lemat}[chapter]
\theorembodyfont{\rm}
\newtheorem{defi}{Definicja}[chapter]
\newtheorem{wni}{Wniosek}[chapter]
\newtheorem{prz}{Przykład}[chapter]
\newenvironment{dowod}{\par\vspace{0.1cm}\par{ \sc Dowód.}}{\hfill $\blacksquare$\par\vspace{0.4cm}\par}
% ----------ustawienia wymiarow strony
\usepackage{geometry}

\newgeometry{tmargin=2.5cm, bmargin=2.5cm, headheight=14.5pt, inner=3cm, outer=2.5cm} 

\linespread{1.1} %-zmiana interlinii

%---------------- Normalne środowiska --------------------
\usepackage{amsmath}

%----------nagłowki i żywa pagina ------------
\pagestyle{fancy} 
%--------------- Wydruk dwustronny
%\cfoot[]{} 
%\lhead[{\scriptsize{\it \thepage}}]{}
%\chead[{\scriptsize\leftmark}]{{\scriptsize \rightmark}}
%\rhead[]{{\scriptsize{\it \thepage}}}
%--------------- Wydruk jednostronny
\fancyhead[C]{} 
\fancyfoot[C]{\thepage}
\fancyhead[L]{\scriptsize\leftmark}
\fancyhead[R]{\scriptsize\rightmark}

\renewcommand{\chaptermark}[1]{%
\markboth{\MakeUppercase{%
\chaptername}\ \thechapter.%
\ #1}{}}

\renewcommand{\sectionmark}[1]{\markright{\thesection.\ #1}}


\usepackage{lipsum} %------ Można usunąć przy pisaniu pracy
%-----------------właściwa część pracy-----------------
\begin{document}
\thispagestyle{empty}

\begin{center}{\sc \Large
Uniwersytet Gdański\\
Wydział Matematyki, Fizyki i Informatyki
}
\end{center}

\vspace{3cm}

\begin{adjustwidth}{120pt}{0pt}
\begin{flushleft}
{\large \textbf{Agnieszka Kamińska\\
Aleksandra Panek\\
Karolina Pomian\\
}
}\par\vspace{0.5cm}\par
{\large
Kierunek: Informatyka Ogólnoakademicka\\
Specjalność: \\
Numery albumu: x, x, 275552\\
}
\end{flushleft}
\end{adjustwidth}

\vspace{1cm}

\begin{center}{ \Large \textbf {
Tytuł pracy
}}
\end{center}

\vspace{3cm}

\begin{flushright}
%praca dyplomowa\\
Praca licencjacka napisana\\
pod kierunkiem…………\\
Hanna Furmańczyk
\end{flushright}
\vfill
\begin{center}{\large
\textbf{Gdańsk 2023}}
\end{center}

\newpage

\vspace*{16cm}
\begin{flushright}
miejsce na ewentualne podziękowania\\
dla promotora
\end{flushright}



\thispagestyle{empty} \setcounter{page}{0} \tableofcontents

\chapter*{Streszczenie: [200-400 słów]}

\thispagestyle{empty} \addcontentsline{toc}{chapter}{Wstęp}

Zakres pracy obejmuje projekt oraz implementację aplikacji webowej – kalkulatora arytmetycznego dla dzieci. Podstawową funkcjonalnością dostarczaną przez aplikację jest wykonywanie prostych obliczeń arytmetycznych takich jak dodawanie i odejmowanie w zakresie dwudziestu, tak aby mogły z niego korzystać dzieci. Ponadto, kalkulator powinien prezentować proces obliczeń w sposób zrozumiały dla dzieci. Aplikacja ma być zgodna ze standardem HTML5, CSS oraz JavaScript.

\chapter{Opis problemu}
\thispagestyle{empty}
\pagestyle{fancy}

[obowiązkowo, dokładny opis, co program robi, do czego służy, jaki jest cel projektu np. proof of concept, prototyp, przetestowanie technologi]

\section{Porównanie dostępnych rozwiązań}

xxx

\section{Możliwości zastosowania praktycznego}

xxx

\chapter{Projekt i analiza}
\thispagestyle{empty}

[obowiązkowo]

\section{Aktorzy i Przypadki użycia}

[obowiązkowo, jakie są podstawowe funkcjonalności]\\
Wymagania funkcjonalne i niefunkcjonalne [obowiązkowo]

\subsection{subsection}
xxx

\section{Diagram klas}

[obowiązkowo]

\section{Diagram modelu danych}

[obowiązkowo, dla projektów korzystających z baz danych relacyjnych - ERD, dla pozostałych opis w jaki sposób dane będą zorganizowane, przechowywane, przykładowe dokumenty dla baz dokumentowych itp.]

\section{Projekt interfejsu użytkownika}

[mogą być makiety, szkice lub gotowe projekty grafik]

\chapter{Implementacja}

[obowiązkowo]

\section{Architektura rozwiązania}

obowiązkowo, diagramy uwzględniające architekturę całości, np. klient-serwer, jakie funkcjonalności na serwerze, a jakie po stronnie klienta, użyte wzorce projektowe np. MVC, RESTFul]

\section{Użyte technologie}

[obowiązkowo, jakie języki programowania, jakie dodatkowe biblioteki, np. na potrzeby mapowania obiektowo-relacyjnego, fraweworki webowe itp.]

\chapter{Testy}

[obowiązkowo w jaki sposób zamierzamy testować, czy automatycznie, czy manualnie i w jakim zakresie ]

\section{Scenariusz testowania}

[jak będziemy testować konkretne funkcjonalności]

\section{Raport z testów}

[tabelka pokazująca jak zaliczono/nie zaliczono scenariusze testowe]

\chapter{Wkład własny}

[obowiązkowo, ogólny podział prac w projekcie, czym konkretnie w projekcie student się zajmował w poszczególnych fazach/etapach projektu]

\begin{thebibliography}{00} %Bibliografia [obowiązkowo, głównie dokumentacja techniczna, książki z inżynierii oprogramowania ale także artykuły z sieci, blogi, nawet wikipedia]
\addcontentsline{toc}{chapter}{Literatura}

%------------- Książka ---------------------
\bibitem{day}
M.M. Day, \textit{Normed Linear Spaces}, Springer-Verlag, Berlin
-- Heidelberg -- New York, 1973.

%------------- Artykuł ---------------------
\bibitem{james} R.C. James, \textit{Orthogonality in normed linear linear spaces}, Duke Math. J., {\bf
12} (1945), 291--301.

\bibitem{k} M. Kuczma, \textit{An Introduction to the
Theory of Functional Equations and Inequalities}, PWN -
Uniwersytet Śląski, Warszawa - Kraków - Katowice, 1985.

%------------- Skrypty, notatki do wykładu ---------------------
\bibitem{poz23}
W. Pleśniak, \  \textit{Wykłady z teorii aproksymacji}, WUJ,
Kraków 2000.

%------------- Strona internetowa ---------------------
\bibitem{OEF}
Numbers of edges of regular polygons constructible with ruler and compass (A003401), The on-line encyclopedia of integer sequences, https://oeis.org/A003401 

\end{thebibliography}
\newpage\thispagestyle{empty}
Imię i Nazwisko \hfill Kraków, dnia wpisać date\par\vspace{1cm}\par
\centerline{{\Large \bf O Ś W I A D C Z E N I E}}
\par\vspace{1.4cm}\par
Świadomy(a) odpowiedzialności oświadczam, że przedłożona praca pt.:
\begin{center}
{\sc Tu wpisać tytuł pracy}
\end{center}
została napisana przeze mnie samodzielnie.
\par
Jednocześnie oświadczam że w/w praca nie narusza praw autorskich w rozumieniu Ustawy z dnia 4 lutego 1994 r. o prawie autorskim i prawach pokrewnych (Dz.U. z 2006 nr 90, poz.
631 z późn. zmianami) oraz dóbr osobistych chronionych prawem cywilnym.
\par
Przedłożona praca nie zawiera danych empirycznych ani też informacji, które uzyskałem(am) w sposób niedozwolony. Stwierdzam, iż przedstawiona praca w całości ani też w części nie była wcześniej podstawą żadnej innej urzędowej procedury związanej z~uzyskaniem dyplomu ani też nadania tytułów zawodowych.
\par\vspace{1cm}\par
\begin{flushright}
....................................... \ \ \ \ \ \ \\
{\scriptsize (podpis)\hspace{2.3cm}\ }
\end{flushright}


\end{document}